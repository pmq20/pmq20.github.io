% !TEX TS-program = xelatex
% !TEX encoding = UTF-8

\documentclass[11pt]{article}

\usepackage{fontspec}
\usepackage{xunicode}
\usepackage{xltxtra}
\usepackage{graphicx}
\usepackage{geometry}
\usepackage{hyperref}
\usepackage{xeCJK}
\usepackage{amsmath}
\usepackage{amssymb}
\usepackage{amsfonts}
\usepackage{mathtools}
\usepackage{zhnumber}
\usepackage{hyperref}
\DeclareMathOperator*{\argmax}{arg\,max}
\DeclareMathOperator*{\argmin}{arg\,min}
\setCJKmainfont{Microsoft YaHei}
\geometry{a4paper}
\linespread{1.2}

\title{评论 Nijkamp 等人 2019 年的《剖析基于马尔可夫链蒙特卡洛的能量模型的最大似然学习》}
\author{潘旻琦}
\date{\zhtoday}

\begin{document}
\maketitle

\begin{enumerate}
\item 虽然传统理论预期了收敛,但 \cite{nijkamp2019anatomy} 发现其实在实践中很难实现收敛;\cite{nijkamp2019anatomy} 的实验非常丰富,但是缺乏理论层面的解释,理论的短板具体在哪里呢?另外 $v_t$和$d_{s_t}$ 两个量的定义是作者的核心关注点,这两个量存在神奇的互相关和自相关性完全是通过实验发现的,没有理论解释,理论上可不可以解释这个现象?除了 $v_t$和$d_{s_t}$ 还可以定义哪些量进行观察?定义新的量或许可以浮现新的有趣现象
\item CNN 能量函数除了 MCMC 还有别的方法取样吗?CNN 可以换成其他的神经网络吗?甚至可以换成无参模型吗?我认为训练能量函数本质上实在模拟一个长期记忆体,或许可以看看神经生物学
人类的大脑是怎么实现记忆的,我猜不一定是一个 CNN 的形状的神经网络;但是换能量函数可能还是得保留可微的特性,否则怎么优化又是个问题
\item \cite{nijkamp2019anatomy} 用的 Langevin 只是众多 MH(Metropolis–Hastings)的一种,朱松纯98年的 FRAME \cite{zhu1998filters} 用的是另外一种 MH,叫 Gibbs Sampler,他的论文 \cite{zhu1998filters} 确实很严谨,各个量都有数学上的定义,非常值得学习;Gibbs Sampler 和 Langevin 的区别在于,Gibbs Sampler 为每个维度分别选择一个新样本(如 \cite{zhu1998filters} 的 Algorithm~2  ),而不是一次为所有维度选择样本,所以在计算上比较昂贵,而 Langevin 采样可以一次性整体采样;例如100张 $3\times32\times32$ 的彩色图片的总计 $307200$ 个维度可以一次性同时更新
\item 计算机视觉领域里人们用 Langevin 采样的原因,我猜是这样的:把图片的每个像素想象成一粒悬浮在水中的花粉,物理学家 Langevin 认为每粒花粉受到两个力的作用:一个粘性阻力,这个力根据斯托克定律是$6\pi\eta r\dot{x}$,在机器学习领域可以类比于能量函数 $U(X;\theta)$ 对 MCMC 的排斥力,一阶导 $\dot{x}$ 对应 $\frac{\partial}{\partial x}U(X;\theta)$,促使下一次随机游走向能量函数低的地方走;另一个力是水分子对花粉的波动的持续冲击力,Langevin 认为是一个零均值的高斯过程,在机器学习领域可以类比于 MCMC 中的蒙特卡洛随机化;这个猜测还需要找更多 Langevin 方面的旧文献来证实
\item 看了源代码后发现 \cite{nijkamp2019anatomy} 的实验的 Langevin 实现是作者手撕的;作者在论文中也提到这个 Langevin 实现并没有进行动量更新和 MH 更新,这样会不会出问题?完整的 Langevin 实现会不会带来不同的结果?
\item \cite{nijkamp2019anatomy} 的实验 \S4.1 有一些未明说的细节,我看了一下源代码,这里记录一下我看到的一些的细节:该实验假定每张图像的维度是 $2\times1\times1$,即只有两个通道、一个像素,且服从真实概率分布 $q(\mathbf{x}): \mathbb{R}^{2\times1\times1}\to[0,\infty)$;这个实验构造了两种多峰分布,一个是八汤圆分布,一个是四环分布;四环分布的定义从源代码反推可知\begin{equation}\label{sihuanfenbu}
q(\mathbf{x})=\frac{1}{8\pi}\cdot\sum_{k=0}^3\frac{1}{k+1}\mathcal{N}\left(\Vert\mathbf{x}\Vert_2;\mu=k+1,\sigma^2=0.15^2\right)
\end{equation}八汤圆分布的定义从源代码反推可知\begin{equation}\label{batangyuan}
q(\begin{bmatrix}x_0\\x_1\end{bmatrix})=\frac{1}{8}\sum_{k=0}^7\mathcal{N}\left(\begin{bmatrix}x_1\\x_0\end{bmatrix};\mu=\begin{bmatrix}\cos(\frac{2\pi k}{8})\\\sin(\frac{2\pi k}{8})\end{bmatrix},\Sigma=\begin{bmatrix}0.15^2&0\\0&0.15^2\end{bmatrix}\right)
\end{equation}所训练的一族有参能量函数叫 $\text{ToyNet}:\mathbb{R}^2\to\mathbb{R}$,其定义从源代码反推可知\[
\text{ToyNet}\equiv C_{64,1}\circ R\circ C_{64,64}\circ R\circ C_{64,64}R\circ C_{32,64}\circ R\circ C_{2,32}
\]其中 $C_{i,j}$ 是一些从$\mathbb{R}^{i\times1\times1}$到$\mathbb{R}^{j\times1\times1}$的$1\times1$卷积仿射变换,$R$ 是一些负斜率恒为 $0.05$ 的 Leaky ReLU 函数
\item \cite{nijkamp2019anatomy} 的实验 \S4.3 也有一些未明说的细节,我看了一下源代码,这里记录一下我看到的一些的细节:该实验假定每张图像的维度是 $3\times32\times32$,即三个彩色通道、每个通道$32\times32$个像素,且服从真实概率分布 $q(\mathbf{x}): \mathbb{R}^{3\times32\times32}\to[0,\infty)$,其定义从源代码反推可知\begin{equation}\label{oxford102flowers}
q(\mathbf{x})=\frac{1}{8189}\sum_{k=1}^{8189}\mathcal{N}\left(\mathbf{x};\mu=\text{Flower}_k,\Sigma=I_{3\times32\times32}\right)
\end{equation}其中 $\text{Flower}_k$ 是牛津102类花数据集的8189张图片的第$k$张的$3\times32\times32$维向量;被训练的能量函数叫 NonlocalNet: $\mathbb{R}^{3\times32\times32}\to\mathbb{R}$,其定义从源代码反推可知\[
\begin{split}
\text{NonlocalNet}(\mathbf{x})&\equiv B_4(B_3(B_2(B_1(\mathbf{x}))))\\
B_1(\mathbf{x})&\equiv \text{MaxPool}(\text{ReLU}(\text{Conv}_{3,32}(\mathbf{x})))\\
B_2(\mathbf{x})&\equiv \text{MaxPool}(\text{ReLU}(\text{Conv}_{32,64}(\text{NonLocalBlock}_{32}(\mathbf{x})))\\
B_3(\mathbf{x})&\equiv \text{MaxPool}(\text{ReLU}(\text{Conv}_{64,128}(\text{NonLocalBlock}_{64}(\mathbf{x})))\\
B_4(\mathbf{x})&\equiv \text{FC}_{256,1}(\text{ReLU}(\text{FC}_{2048,256}(\mathbf{x})))
\end{split}
\]其中 NonLocalBlock 是两年前 \cite{wang2018non} 提出的,其想法来源于十五年前的 \cite{buades2005non};\cite{buades2005non} 当时发现只用局部平滑滤波器做图像降噪的效果不如 NL-means好;NL-means 的方法是根据图像全局的自相似性加权的图像中所有像素的平均;\cite{wang2018non} 则把 NL-means 包装成一个可插拔的神经网络层
\item \cite{nijkamp2019anatomy} 的实验 \S4.2 用的能量函数叫做 VanillaNet,论文里没写它的构造,这里记录一下我通过源代码反推的构造:\[
C_{256,1} \circ R \circ C_{128,256} \circ R \circ C_{64,128} \circ R \circ C_{32,64} \circ R \circ C_{3,32}
\]其中 $R$ 是一些负斜率恒为 $0.05$ 的 Leaky ReLU 函数,其他 $C$ 的定义如下:
\begin{enumerate}
\item $C_{3,32}:\mathbb{R}^{3\times32\times32}\xrightarrow{3\times3\text{ 卷积核,步伐1,填充1}}\mathbb{R}^{32\times32\times32}$
\item $C_{32,64}:\mathbb{R}^{32\times32\times32}\xrightarrow{4\times4\text{ 卷积核,步伐2,填充1}}\mathbb{R}^{64\times16\times16}$
\item $C_{64,128}:\mathbb{R}^{64\times16\times16}\xrightarrow{4\times4\text{ 卷积核,步伐2,填充1}}\mathbb{R}^{128\times8\times8}$
\item $C_{128,256}:\mathbb{R}^{128\times8\times8}\xrightarrow{4\times4\text{ 卷积核,步伐2,填充1}}\mathbb{R}^{256\times4\times4}$
\item $C_{256,1}:\mathbb{R}^{256\times4\times4}\xrightarrow{4\times4\text{ 卷积核,步伐1,填充0}}\mathbb{R}^{1\times1\times1}$
\end{enumerate}
\item 复现了 \cite{nijkamp2019anatomy} 的实验 \S4.1,用 Langevin 噪音 $\varepsilon=0.125$、MCMC 步数 $L=500$ 训练 ToyNet,让它学习分布 (\ref{sihuanfenbu}),每批训练100个样本,我的实验结果如下:
\begin{enumerate}
\item 早在31000批训练之后,持久初始化五百步 MCMC 短跑取负样本在核密度估计下的图案就可以显现出四环的形状,而此时 CNN 的输出在$\mathbb{R}^2$上的图像还没有完全收敛到四环的形状,这大致符合作者的结论——短跑出样本容易,长跑收敛难
\item 约93000批之后 CNN 在$\mathbb{R}^2$上的输出所作的图像才基本接近真实分布的四环的形状,说明 CNN 成功学习到了真实分布;最终跑到二十万批,中间学习到的分布略有扰动,但整体没有太多的偏离,基本可以认为实现了收敛的最大似然学习
\end{enumerate}
\item 复现了 \cite{nijkamp2019anatomy} 的实验 \S4.2,用 Langevin 噪音 $\varepsilon=0.01$、MCMC 步数 $L=150$ 训练 VanillaNet,让它学习分布 (\ref{oxford102flowers}),每批训练100张图,跑完了十万批,我的实验结果如下:
\begin{enumerate}
\item 用数据初始化 MCMC 进行十万步长跑,在训练完一万批、两万批……九万批、十万批之后进行 MCMC 长跑,均得到非常糟糕的结果,基本都只有一整块、一整块的颜色块,什么都看不出来,说明学习到的分布没有收敛到真实分布
\item 用随机噪声初始化 MCMC 一百五十步短跑,100批之后短跑就能得到隐约的花的形状,8800批之后短跑就能得到非常好看的图了,之后的九万多批训练完了之后基本没有改善;短跑出来的图始终带有一些椒盐颗粒感,这是可以理解的,因为毕竟是从噪音初始化来的而且只是短跑,跑到山腰上就停下来了;而且用的能量函数还是这么简单的 VanillaNet,说明只出好图不收敛的 EBM 最大似然学习真的非常容易
\item 进一步观察 $d_{s_t}$和$r_t$,$d_{s_t}$确实在0附近震荡;$d_{s_t},r_t$都表现出短时滞上的强烈的负自相关性,$d_{s_t},r_t$也确实呈现出扩张、伸缩相关性,说明该实验在文中第一个轴上的表现是良好的,而在第二个轴上的表现不佳 
\end{enumerate}
\item 把 \cite{nijkamp2019anatomy} 的实验 \S4.3 (Langevin 噪音 $\varepsilon=0.0075$、MCMC 步数 $L=500$ 训练 NonlocalNet,让它学习分布 (\ref{oxford102flowers}),每批训练100张图)训练到了二十万批,我的实验结果如下:
\begin{enumerate}
\item 若用五百步 MCMC (用持久初始化)在学习到的分布上取样,那么在仅仅训练了100批之后,五百步 MCMC 跑出来的负样本就已经有模糊的花的样子出现了,且带有一些雪花;经过3300批之后基本上五百步 MCMC 跑出来的负样本都有视觉上可识别的花出现;一万批之后不再有雪花;五百步这么短的 MCMC 跑出来的负样本都是视觉效果良好的花,于是复现了作者的结论——MCMC 短跑很容易跑出好看的图
\item 若用十万步 MCMC (用数据初始化而不是持久初始化)在学习到的分布上取样,那么一万批至四万批训练之后长跑得到的都是过饱和的图,直到五万批训练之后 MCMC 长跑才得到一个视觉上略佳的结果,于是复现了作者的结论——MCMC 长跑很难跑出好看的图,最大似然学习要收敛是很难的
\item 持久初始化后的 MCMC 会随着批次的增加渐渐向真实分布 $q$ 的山峰方向行走(视觉上体现为雪花越来越少),且在三千批之后就基本稳定(视觉上体现为只有微弱的噪音级别的变化);长期跑的持久初始化的 MCMC 初值出现了跨越山峰的行为,例如十几万批之后一些MCMC初值的花的样子变了,而且目测似乎变到了离之前的山峰不太远的另外一个山峰
\item 十万批、十一万批……二十万批后用数据初始化的十万步 MCMC 跑出来的效果都很好,没有出现任何退化,说明学习到的分布应该是真的收敛到真实分布了
\item 若用五百步 MCMC (用持久初始化)在学习到的分布上取样还是一如既往的好,说明收敛的 \S4.3 至少达到了不收敛的 \S4.2 同样好的效果,而且对比实验 \S4.2 用噪声初始化 \S4.3 在视觉上的椒盐颗粒感稍微弱一些
\item $d_{s_t}$确实在0附近震荡;$r_t$ 最终收敛到约$0.06$,五万批至二十万批之间基本没有偏离这个均值,看上去 Langevin 确实梯度成功平衡住了 Langevin 噪音
\end{enumerate}
\end{enumerate}

\renewcommand\refname{参考文献}
\bibliographystyle{apalike}
\bibliography{../../bibliography.bib}

\end{document}
